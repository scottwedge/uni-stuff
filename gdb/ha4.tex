\documentclass[a4paper,12pt]{scrartcl}

\usepackage[utf8]{inputenc}
\usepackage[ngerman]{babel}
\usepackage{scrpage2}\pagestyle{scrheadings}
\usepackage{graphicx}
\usepackage{pgfplots}
\usepackage{multicol}
\usepackage{tikz}
\usepackage{amssymb}
\usepackage{amsmath}
\usepackage{ulem}

\ihead{Aufgabenblatt 5}
\ohead{\today}
\chead{Gruppe Dammer, Teuteberg, Wilhelm}
\pagestyle{scrheadings}

\begin{document}

\section{Aufgabe 1}
* L\"osung *

\section{Aufgabe 2}
*L\"osung*

\section{Serialisierbarkeit und Anomalien} 	
\begin{enumerate}
\subsection*{S1}
	\item[a)]
	\(S_1\) = \(r_1\)(B)\(r_1\)(A)\(w_1\)(A)\(r_2\)(A)\(w_2\)(B)\(w_2\)(A)\\
	1: \(r_1\)(B)  B = 10\\
	2: \(r_1\)(A)  A = 5\\
	3: \(w_1\)(A)  A = A + 180 + B = 195\\
	4: \(r_2\)(A)  A = A ge\"anderte = 195\\
	5: \(w_2\)(B)  B = alte A = 195\\
	6: \(w_2\)(A)  A = A + 110 = 195 + 110 = 305\\
	A = 305\\
	B = 195
	
	\item[b)]
	\(T_1\) wird komplett vor \(T_2\) durchgeführt.
	
	\item[c)]
	Seriell, denn \(T_2\) wird vollkommen vor \(T_1\) ausgeführt.
	
\subsection*{S2}
	\item[a)]
	A = 195\\
	B = 10
	\item[b)]
	\(T_1\) überschreibt alle Änderungen, die \(T_2\) gemacht hat: \\
	zuerst wurde A und B von \(T_1\) und \(T_2\) gelesen (die Werte sind gleich), aber \(T_1\) wird nach \(T_2\) mit alten Werten durchgeführt. 
	\item[c)]
	Hier ist Anomalie "Lost Update" getroffen. Nicht serialisierbar, denn \(T_1\) überschreibt die Änderungen von \(T_2\) vollständig. 
	
\subsection*{S3}
	\item[a)]
	A = 300\\
	B = 5
	\item[b)]
	\(T_1\) beachtet alle Änderungen, die \(T_2\) gemacht hat. Zuerst gehen write-Operationen von \(T_2\) und dann read-Operationen von \(T_1\).
	\item[c)]
	Serialisierbar, denn die Änderungen der Transaktionen werden untereinander berücksichtigt.
	
\subsection*{S4}
	\item[a)]
	A = 200\\
	B = 10 
	\item[b)]
	\(T_1\) beachtet von \(T_2\) gemachte B-Wert Änderung, aber schreibt A-Wert komplett.
	\item[c)]
	Hier wird teilweise (nur A wird überschrieben) Anomalie "Lost Update" getroffen. Nicht serialisierbar, denn Änderungen gehen verloren.
	
\subsection*{S5}
	\item[a)]
	A = 120\\
	B = 10	
	\item[b)]
	\(T_2\) liest die Werte vor bei \(T_1\) gemachte Änderungen.  
	\item[c)]
	Die Änderungen gehen verloren, deswegen nicht serialisierbar (Update Lost).
		
\subsection*{S6}
	\item[a)]
	A = 380\\
	B = 10
	\item[b)]
	\(T_1\) wird nach der \(T_2\) durchgeführt und berücksichtigt alle Änderungen.
	\item[c)]
	Serialisierbar, denn die Änderungen der Transaktionen werden untereinander berücksichtigt.
	
\end{enumerate}


\end{document}