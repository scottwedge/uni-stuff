\documentclass[ngerman]{gdb-aufgabenblatt}


\renewcommand{\Aufgabenblatt}{3}
\renewcommand{\Ausgabedatum}{Mi. 16.10.2013}
\renewcommand{\Abgabedatum}{Fr. 01.11.2013}
\renewcommand{\Gruppe}{Dammer, Teuteberg, Wilhelm}


\begin{document}


\section{Beispiel f\"ur ER-Diagramm}

\begin{center}
\begin{tikzpicture}

\node[entity] (e1) {E1};
\node[attribut] (e1-a1) [above left =5mm and 4mm of e1.north] {\underline{A1}} edge (e1);
\node[attribut] (e1-a2)  [above right=5mm and 4mm of e1.north] {A2} edge (e1);
\node[multivalentattribut] (e1-a3)  [right=5mm of e1] {A3} edge (e1);

\node[entity] (e2) [below left =1cm and 1mm of e1.south] {E1} edge [erbt] (e1);
\node[entity] (e3)      [below right=1cm and 1mm of e1.south] {E3} edge [erbt] (e1);
\node[entity] (e4) [right =7cm of e3] {E4};

\node[weakentity] (e5) [below =3cm of e3] {E5};
\node[attribut] (e5-a1)  [right=5mm of e5] {\dashuline{A1}} edge (e5);

\node[relationship] (r1) [right=2cm of e3] {R1};
\path (r1) edge node[at end,anchor=north west] {$[0;2]$} (e3);
\path (r1) edge node[at end,anchor=north east] {$[7;9]$} (e4);


\node[weakrelationship] (r2) [below=1cm of e3] {R2};
\path (r2) edge node[at end,anchor=north west] {$1$} (e3);
\path (r2) edge node[at end,anchor=south east] {$8$} (e5);



\end{tikzpicture}
\end{center}






\section{Logischer Entwurf}

\begin{RMSchma}
Film(\soliduline{Titel}, \dashuline{Name $\rightarrow$ Regisseur.Name}, Zusammenfassung, 1.Drehtag, letzter Drehtag)

Charakter(\soliduline{CID}, Name, Charakterbeschreibung)

Genre(\soliduline{Name})

Regisseur(\soliduline{Name}, \dashuline{Name $\rightarrow$ Genre.Name}) oder Regisseur(\soliduline{\dashuline{Name $\rightarrow$ Person.Name}}, \dashuline{Name $\rightarrow$ Genre.Name})

Schauspieler(\soliduline{Name}) oder Schauspieler(\soliduline{\dashuline{Name $\rightarrow$ Person.Name}})

Person(\soliduline{Name}, DOB,Geschlecht)

Markenzeichen(\soliduline{\dashuline{Name $\rightarrow$ Schauspieler.Name}, Zeichen})

geh\"ort zu(\soliduline{\dashuline{Titel $\rightarrow$ Film.Titel}, \dashuline{Name $\rightarrow$ Genre.Name}})

spielt(\soliduline{\dashuline{Name $\rightarrow$ Schauspieler.Name}, \dashuline{CID $\rightarrow$ Charakter.CID}, \dashuline{Titel $\rightarrow$ Film.Titel}}, Drehbeginn, Drehende, Gage)

\end{RMSchma}






\section{Relationale Algebra und SQl}

\begin{align*}
  \item[a)]
  \begin{enumerate}
   \item [i)] Finde alle (Namen) Rennfahrer, die 1 Platz in ``Malaysia GP'' genommen haben.
   \item[ii)] Finde alle Rennfahrer, die in Rennstall mit Budget kleiner 350 teilgenommen haben.
   \item[iii)] Finde alle Teams (Rennstall), deren Rennfahrern ein Paltz in ``Australian GP" genommen haben.
  \end{enumerate}
  \item[b)]
  \begin{enumerate}
   \item[i)] &\projektion{Rennstall.Name}(\selektion}{Geburt>=1985}(Rennfahrer \verbund Rennstall))
   \\ &=\{ \wert(Red Bull), \wert(McLaren) \} 
   
   \item[ii)] &\projektion{Voraname,Nachname,Geburt}((\selektion{Rennstall=31}Rennfahrer) \verbund \selektion{OID=4}Platzierung) 
   \\ &=\{ \wert(Lewis Hamilton 1985-01-07), \wert(Jenson Button 1980-01-19) \}
   
   \item[iii)] &(Rennfahrer)-(\projektion{RID,Vorname,Nachname,Geburt,Wohnort,Rennstall}(Rennfahrer \verbund Platzierung)
   \\ &= \{ \wert(44 Kimi R\'aikk\'onen 1979-10-17 Espoo(Finnland) 34) \}
   
   \item[iv)] &\projektion{Vorname,Nachname}(\selektion{Rennstall=\projektion{Rennstall}(\selektion{RID=20}Rennfahrer)})
   \\ &=\{ \wert(Lewis Hamilton) \}
  \end{enumerate}
  \item[c)]
  \begin{enumerate}
   \item[i)] 
   \begin{verbatim}
    SELECT
      rf.Vorname, rf.Nachname, rf.Geburt
    FROM
      Rennfahrer rf,
      Platzierung pl,
    WHERE
      pl.OID = 4 AND
      rf.Rennstall = 31
   \end{verbatim}
   
   \item[ii)]
   \begin{verbatim}
    SELECT
      Vorname, Nachname
    FROM
      Rennfahrer
    WHERE
      Rennfahrer.Rennstall = 31 AND
      Rennfahrer.Vorname != "Button"
   \end{verbatim}
  \end{enumerate}
\end{align*}

\section{Algebraische Optimierung}
\begin{enumerate}
 \item[a)]
 \begin{tikzpicture}
 \node (Rennfahrer) {Rennfahrer};
 \node (Platzierung) [right=25mm of Rennfahrer] {Platzierung};
 \node (Rennort) [right=27mm of Platzierung] {Rennort};
 \node (j1) [above=20mm of $(Rennfahrer)!.5!(Platzierung)$] {$\verbund$};
 \node (j2) [above=22mm of j1] {$\verbund$};
 \node (sigma1) [above=of j2] {$\selektion{Platz<11}$};
 \node (sigma2) [above=of sigma1] {$\selektion{Name="China GP"}$};
 \node (pi1) [above=of sigma2] {$\projektion{Wohnort}$};
 \node (final) [above=of pi1] {};
 
 \path (Rennfahrer) edge node[smallr,near start,above right] {?? Tupel\\?? Attribute} (j1);
 \path (Platzierung) edge node[smallr,near start,above left] {?? Tupel\\?? Attribute} (j1);
 \path (j1) edge node[smallr,near start,above right] {?? Tupel\\?? Attribute} (j2);
 \path (Rennort) edge node[smallr,near start,above left] {?? Tupel\\?? Attribute} (j2);
 \path (j2) edge node[smallr,midway,left] {$??\cdot\frac{??}{??}=??$ Tupel\\?? Attribute} (sigma1);
 \path (sigma1) edge node[smallr,midway,left] {$??\cdot\frac{??}{??}=??$ Tupel\\?? Attribute} (sigma2);
 \path (sigma2) edge node[smallr,midway,left] {$??\cdot\frac{??}{??}=??$ Tupel\\?? Attribute} (pi1);
 \path (pi1) edge node[smallr,midway,left] {$??\cdot\frac{??}{??}=??$ Tupel\\?? Attribute} (final);
 
 \end{tikzpicture}
 
 \item[b)]
 \begin{tikzpicture}
 \node (Rennfahrer) {Rennfahrer};
 \node (Platzierung) [right=25mm of Rennfahrer] {Platzierung};
 \node (Rennort) [right=27mm of Platzierung] {Rennort};
 \node (sigma1) [above=20mm of Platzierung] {$\selektion{Platz<11}$};
 \node (sigma2) [above=20mm of Rennort] {$\selektion{Name="China GP"}$};
 \node (j1) [above=20mm of $(sigma1)!.5!(sigma2)$] {$\verbund$};
 \node (j2) [above=20mm of $(Rennfahrer)!.5!(j1)$] {$\verbund$};
 \node (pi1) [above=20mm of j2] {$\projektion{Wohnort}$};
 
 \path (Platzierung) edge node[smallr,midway,left] {$??\cdot\frac{??}{??}=??$ Tupel\\?? Attribute} (sigma1);
 \path (Rennort) edge node[smallr,midway,left] {$??\cdot\frac{??}{??}=??$ Tupel\\?? Attribute} (sigma2);
 \path (sigma1) edge node[smallr,near start,above right] {?? Tupel\\?? Attribute} (j1);
 \path (sigma2) edge node[smallr,near start,above left] {?? Tupel\\?? Attribute} (j1);
 \path (Rennfahrer) edge node[smallr,near start,above right] {?? Tupel\\?? Attribute} (j2);
 \path (j1) edge node[smallr,near start,above left] {?? Tupel\\?? Attribute} (j2);
 \path (j2) edge node[smallr,midway,left]{$??\cdot\frac{??}{??}=??$ Tupel\\?? Attribute} (pi1);
 
 \end{tikzpicture}
 \item Ausdruck b) hat den höchsten Optimierungsgrad: Nach der Optimierungsheuristik I, 
 werden Selektionen möglichst früh ausgeführt, nach II werden die Projektionen möglichst früh ausgeführt und nach III 
 werden Operatoren Selektion und Projektion verknüpft. In Ausdruck a) werden erst die Relationen verknüpft und dann 
 die Selektionen und Projektionen durchgeführt, das ist nicht effizient. \\
  

\end{enumerate}
\end{document}